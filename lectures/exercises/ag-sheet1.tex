\documentclass[12pt,a4paper]{article}

\usepackage{amsmath,amssymb,latexsym}
\usepackage{stmaryrd}
\usepackage[utf8]{inputenc}
\usepackage[all]{xy}
\usepackage{fancybox}
\usepackage{german}
\usepackage{paralist}
\usepackage{enumitem,leftidx}
\usepackage{ifthen}
\usepackage{fancyhdr}
\setenumerate[1]{label=(\alph*)}
\setenumerate[2]{label=(\roman*)}

\usepackage{a4wide}

\parindent=0pt


\newtheorem{theorem}{Theorem}%[section]
\newtheorem{lemma}[theorem]{Lemma}

\begin{document}
\pagestyle{fancy}

\vspace{-.5cm}
\centerline{ \textsc{\Large PMH2: Commutative Algebra} }
\centerline{ \textsc{University of Sydney, 2018}}

\vspace{.3cm}

\begin{center}
  {\large\bf Assignment 1}\\
\end{center}


\begin{center}
Let $R$ denote an associative, commutative, and unital ring.
\end{center}

\paragraph{Exercise 1. (4 points)}
Let $S$ be a ring, $f:R \longrightarrow S$ a map of rings and $I < R$ respectively $J < S$ ideals. We denote by $I^\mathrm{e} = (f(I)) < S$ the \emph{extension} of $I$ (along $f$) and by $J^\mathrm{c}=f^{-1}(J) < R$ the \emph{contraction} of $J$ (along $f$). Show the following statements for extensions and contractions of ideals along the map $f$.
\begin{enumerate}
\item $I \subset {\left(I^{\mathrm{e}}\right)}^\mathrm{c}$ and ${\left(J^{\mathrm{c}}\right)}^\mathrm{e} \subset J$.
\item $I^\mathrm{e} = \left({\left(I^{\mathrm{e}}\right)}^\mathrm{c}\right)^\mathrm{e}$ and $J^\mathrm{c} = \left({\left(J^{\mathrm{c}}\right)}^\mathrm{e}\right)^\mathrm{c}$.
\item Denote by $\mathcal{C}$ the set of ideals in $R$ obtained as contractions of ideals in $S$ and by $\mathcal{E}$ the set of ideals in $S$ obtained as extensions of ideals in $R$. Then it holds
\[
\mathcal{C}=\{ I < R \mid I = {\left(I^{\mathrm{e}}\right)}^\mathrm{c} \}\text{ and } \mathcal{E}=\{ J < S \mid J = {\left(J^{\mathrm{c}}\right)}^\mathrm{e}\}\\
\]
and
\begin{eqnarray*}
\mathcal{C}&\longleftrightarrow& \mathcal{E}\\
I & \longmapsto & I^\mathrm{e}\\
J^\mathrm{c} & \longmapsfrom & J
\end{eqnarray*}
is a one-to-one correspondence.
\end{enumerate}

\paragraph{Exercise 2. (4 points)}
Denote by $R[x]$ the polynomial ring in one indeterminant and coefficients in $R$. Let $f = r_0 + r_1 x + r_2 x^2 + \ldots + r_n x^n \in R[x]$. Show that
\begin{enumerate}
\item $f$ is a unit in $R[x]$ if and only if $r_0$ is a unit in $R$ and $r_1,\ldots,r_n$ are nilpotent in $R$.
\item $f$ is nilpotent in $R[x]$ if and only if $r_0,\ldots,r_n$ are nilpotent in $R$.
\item $f$ is a zero-divisor in $R[x]$ if and only if there exists $r \in R \setminus \{0\}$ such that $rf=0$.
\end{enumerate}

\paragraph{Exercise 3. (4 points)}
Let $R$ be an associative and unital ring, but not necessarily commutative, such that $x^2 = x$ for every $x \in R$.
\begin{enumerate}
\item Show that $x+x=0$ for all $x \in R$.
\item Show that $R$ is commutative.
\item Show that any finitely generated ideal in $R$ is a principal ideal.
\item Show that any prime ideal of $R$ is a maximal ideal.
\end{enumerate}

\paragraph{Exercise 4. (4 points)}
Let $f:R \rightarrow S$ be a surjective map of rings and
\begin{align*}
\left\lbrace \begin{array}{c}
\text{ideals of $S$}
\end{array}
\right\rbrace
&\begin{array}{c} \overset{\Phi}{\longrightarrow} \\ \underset{\Psi}{\longleftarrow}
\end{array}
\left\lbrace \begin{array}{c}
\text{ideals of $R$} \\ \text{containing } \mathrm{ker}(f)
\end{array}
\right\rbrace.
\end{align*},
where $\Phi(I) = f^{-1}(I)$ and $\Psi(J)=f(J)$.
\begin{enumerate}
\item Show that $\Phi$ and $\Psi$ define inclusion preserving bijections that are mutually inverse to each other.
\item Show that this can be restricted to prime ideals in $S$ on the left hand side and prime ideals containing $\mathrm{ker}(f)$ in $R$ on the right hand side.
\end{enumerate}

\paragraph{Exercise 5. (4 points)}
\begin{enumerate}
\item Let $\{M_i\}_{i \in  I}$ and $\{N_i\}_{i \in I}$ be families of $R$-modules and $\{ f_i : M_i \rightarrow N_i \}$ a family of $R$-module maps. Show that this naturally determines maps 
\[
f_\oplus : \bigoplus_{i \in I} M_i \rightarrow \bigoplus_{i \in I} N_i \quad \text{and} \quad f_\Pi : \prod_{i \in I} M_i \rightarrow \prod_{i \in I} N_i.
\]
\item Let $\{M_i\}_{i \in I}$ be a family of $R$-modules and $N$ an $R$-module. Show that
\begin{eqnarray*}
\mathrm{Hom}_R\left(\bigoplus_{i\in I} M_i,N \right) &\cong& \prod_{i \in I} \mathrm{Hom}_R \left(M_i,N \right) \text{ and }\\
\mathrm{Hom}_R\left(N,\prod_{i\in I} M_i \right) &\cong& \prod_{i \in I} \mathrm{Hom}_R \left(N,M_i \right)
\end{eqnarray*}
\emph{Hint:} Use the universal properties of direct sum and direct product to show the existence of maps in a suitable direction as well as their injectivity and surjectivity.
\end{enumerate}


\end{document} 